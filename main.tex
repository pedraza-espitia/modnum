\documentclass{article}
\usepackage[utf8]{inputenc}
\usepackage[spanish,es-nodecimaldot,es-tabla]{babel}
\usepackage{amsmath}
\usepackage{graphicx}
\usepackage[colorlinks=true, allcolors=blue]{hyperref}
% hyperref para autoref, amsmath para split
\graphicspath{{./figs/}{./imgs/}}
\usepackage[font=small,labelfont=bf]{caption}
\usepackage{listings}

\title{Tarea 3}
\author{Pedraza-Espitia Salvador}
\date{}

\begin{document}

\maketitle

\section{Corte meridional}
Realizar un programa que realice y grafique un corte meridional de U en donde se observe el
paso del huracán y guardar las gráficas de cuatro tiempos diferentes en formato jpg.


\lstinputlisting[language=Matlab]{./MatlabCodes/inciso1.m}
\href{img.jpg}{clic aquí para abrir script}
\url{google.com}

\section{Divergencia}
Hacer un programa que calcule la divergencia del viento a 10 metros, utilizando diferencias
finitas centradas, sin emplear funciones de matlab (DIV), y guardar las gráficas resultantes de
cuatro tiempos diferentes. Tomar en cuenta que la latitud y longitud que proporciona el
modelo están en grados y deberán convertirse a metros.

\section{Rotacional}
Hacer un programa que calcule el rotacional del viento en el nivel 8, utilizando diferencias
finitas centradas, sin emplear funciones de matlab (CURL), y guardar las gráficas resultantes
de cuatro tiempos diferentes. Tomar en cuenta que la latitud y longitud que proporciona el
modelo están en grados y deberán convertirse a metros.

\section{Resolver \texorpdfstring{$y' = cos(t)+sin(t)$}{y' = cos(t)+ ...} usando RK4}
Considera la ecuación diferencial $y' = cos(t) +sin(t)$ con $y(t=0) = 1$ y $h=0.1$. Escribe un código
para calcular la solución con el método de Runge-Kutta de orden 4.
Recuerda que:
\begin{equation}
    \begin{split}
    k_1 = & h*f(t_i, y_i)\\
    k_2 = & h*f(t_i + h/2, y_i + k_1/2)\\
k_3 = & h*f(t_i + h/2, y_i + k_2/2)\\
k_4 = & h*f(t_i + h, y_i + k_3)
    \end{split}
\end{equation}
\begin{equation}
    y_{i+1} = y_i + 1/6(k_1 + 2k_2 + 2k_3 + k_4)
    \label{eq:rec}
\end{equation}
\section{Resolver \texorpdfstring{$y'=y- 2*t^3 + 2$}{y'=y  2*t ...} con RK4}
Considera la ecuación diferencial $y' = y- 2*t^3 + 2$ con $y(t=0) = 0.5$. Escribe un código para
calcular la solución con el método de Runge-Kutta de orden 4.

optimizar y proteger pdf al final.

Un archivo .m con secciones %%
para cada inciso,
y archivos .txt para cada inciso.
\end{document}