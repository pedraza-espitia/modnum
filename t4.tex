\documentclass{article}
\usepackage[utf8]{inputenc}
\usepackage[spanish,es-nodecimaldot,es-tabla]{babel}
\usepackage{amsmath}
\usepackage{graphicx}
\usepackage[colorlinks=true, allcolors=blue]{hyperref}
\usepackage[makeroom]{cancel}
% hyperref -->autoref, amsmath -->split
\usepackage{subfig,placeins}
\usepackage{libertine}
\usepackage[libertine]{newtxmath}
\graphicspath{{./figs/}{./imgs/}}
\usepackage[font=small,labelfont=bf]{caption}
\usepackage{listings,figs/tuneatantito}
\newcommand\pder[2]{\ensuremath
{\dfrac{\partial#1}{\partial#2}}} 
\newcommand{\ppder}[2]{ \ensuremath {\dfrac{\partial^2
#1}{\partial #2^2}}}
\newcommand{\ppcder}[3]{ \ensuremath {\dfrac{\partial^2
#1}{\partial #2\partial #3}}}

\title{Tarea 4}
\author{\href{https://git.io/salvador}{Pedraza-Espitia S.}}
\date{}

\begin{document}

\maketitle

\section{Circulación atmosférica y circulación oceánica}
Componentes de los modelos

\section{Modelo de circulación atmosférica}

\section{Elípticas, parabólicas o hiperbólicas}

\section{... criterio de clasificación}

\section{Criterio de Courant-Friedrichs-Lewy}

\newcommand\derord[3][]{
\def\temp{#1}\ifx\temp\empty
	\ensuremath {\left( \dfrac{\mathrm{d}#2}{\mathrm{d}#3}}
	\right)_j%
\else
	\ensuremath {\left(
	\dfrac{\mathrm{d}^#1#2}{\mathrm{d}#3^#1}} \right)_j%
\fi}
%\newcommand\dero[3]{\ensuremath {\left( \dfrac{\mathrm{d}^#3#1}{\mathrm{d}#2^#3}} \right)_j}
\newcommand\Dt{\ensuremath {\left( \Delta t \right)}}
\newcommand\Dx{\ensuremath {\left( \Delta x \right)}}
\newcommand\dDx{\ensuremath {\left( 2\Delta x \right)}}
\newcommand\dxdero[3]{\ensuremath {\Dt^{\the\numexpr #3 - 1\relax} \derord[#3]{#1}{#2} }}

\section{Aproximación de la derivada en diferencias hacia adelante, hacia atrás y centradas}

\section{El esquema numérico de diferencias finitas hacia atrás es de orden 1}

%%%%%%%%
\begin{equation}
x_{j-1} = x_j - \Dt\derord{x}{t} + \frac12\Dt^2\derord[2]{x}{t}
- \frac1{3!}\Dt^3\derord[3]{x}{t} + \cdots
\end{equation}

\begin{equation}
\frac{x_{j-1} - x_j}{\Dt} = - \derord{x}{t} +
\frac12\Dt\derord[2]{x}{t} - 
	\frac1{3!}\dxdero{x}{t}{3} + \cdots
\end{equation}

\begin{equation}
\frac{x_{j} - x_{j-1}}{\Dt} = \derord{x}{t} -
\frac12\Dt\derord[2]{x}{t} + \frac1{3!}\dxdero{x}{t}{3} -
\cdots
\end{equation}

\begin{equation}
\epsilon = - \frac12\Dt\derord[2]{x}{t} + \frac1{3!}\dxdero{x}{t}{3} - \cdots
\end{equation}

\begin{equation}
\epsilon = O\Dt
\end{equation}
%%%%%%%%%

\section{Ecuación de advección con el esquema de diferencias centradas}
\lstinputlisting[language=Matlab]{./MatlabCodes/t4inciso8.m}
%\input{inputs/t4i9}

\section{Esquema de salto de rana y estabilidad}
\lstinputlisting[language=Matlab]{./MatlabCodes/t4inciso9.m}

\section{Esquema de diferencias centradas con \texorpdfstring{$O(\Delta x)$}{} error de orden 4}

% u_{j+1}
\begin{equation}
u_{j+1} = u_j + \Dx\derord{u}{x} + 
\frac12\Dx^2\derord[2]{u}{x} + 
\frac1{3!}\Dx^3\derord[3]{u}{x} + \cdots
\label{eq:ujp1}
\end{equation}

% u_{j-1}
\begin{equation}
u_{j-1} = u_j - \Dx\derord{u}{x} + 
\frac12\Dx^2\derord[2]{u}{x} - 
\frac1{3!}\Dx^3\derord[3]{u}{x} + \cdots
\label{eq:ujm1}
\end{equation}

\eqref{eq:ujp1} - \eqref{eq:ujm1}:
\begin{equation}
u_{j+1} - u_{j-1} = 0 + 2\Dx\derord{u}{x} + 0 +
\frac2{3!}\Dx^3\derord[3]{u}{x} + 0 +
\frac2{5!}\Dx^5\derord[5]{u}{x} + \cdots
\label{eq:ujp1mujm1}
\end{equation}

\begin{equation}
\frac{u_{j+1} - u_{j-1}}{2\Dx} = \derord{u}{x} +
\frac1{3!}\Dx^2\derord[3]{u}{x} +
\frac1{5!}\Dx^4\derord[5]{u}{x} + \cdots
\end{equation}


% u_{j+2}
\begin{equation}
u_{j+2} = u_j + \dDx\derord{u}{x} +
\frac12\dDx^2\derord[2]{u}{x} +
\frac1{3!}\dDx^3\derord[3]{u}{x} + \cdots
\label{eq:ujp2}
\end{equation}

% u_{j-2}
\begin{equation}
u_{j-2} = u_j - \dDx\derord{u}{x} +
\frac12\dDx^2\derord[2]{u}{x} -
\frac1{3!}\dDx^3\derord[3]{u}{x} + \cdots
\label{eq:ujm2}
\end{equation}

\eqref{eq:ujp2} - \eqref{eq:ujm2}:
\begin{equation}
u_{j+2} - u_{j-2} = 0 + 2\dDx\derord{u}{x} + 0 +
\frac2{3!}\dDx^3\derord[3]{u}{x} + 0 +
\frac2{5!}\dDx^5\derord[5]{u}{x} + \cdots
\label{eq:ujp2mujm2}
\end{equation}

\begin{equation}
\frac{u_{j+2} - u_{j-2}}{2\dDx} = \derord{u}{x} +
\frac1{3!}\dDx^2\derord[3]{u}{x} +
\frac1{5!}\dDx^4\derord[5]{u}{x} + \cdots
\label{eq:ujp2mujm2_s}
\end{equation}

El objetivo es eliminar el término con orden cuadrático
$\dDx^2\derord[3]{u}{x}$, 
necesitamos los coeficientes apropiados


\begin{equation}
\def\coef{\frac43}
\coef\frac{u_{j+1} - u_{j-1}}{2\Dx} = \coef\derord{u}{x} +
\coef\frac1{3!}\Dx^2\derord[3]{u}{x} +
\coef\frac1{5!}\Dx^4\derord[5]{u}{x} + \cdots
\label{eq:4tercios}
\end{equation}

\begin{equation}
\def\coef{\frac13}
\coef\frac{u_{j+2} - u_{j-2}}{2\dDx} = \coef\derord{u}{x} +
\coef\frac1{3!}\dDx^2\derord[3]{u}{x} +
\coef\frac1{5!}\dDx^4\derord[5]{u}{x} + \cdots
\label{eq:1tercio}
\end{equation}

\eqref{eq:4tercios} - \eqref{eq:1tercio}:
\begin{equation}
\def\u{\frac13}
\def\c{\frac43}
    \begin{split}
\c\frac{u_{j+1} - u_{j-1}}{2\Dx}-\u\frac{u_{j+2} -
u_{j-2}}{2\dDx} = & \derord{u}{x} + \cancel{ \left[
\c\frac{1}{3!} - \u\frac{1}{3!}(2)^2 \right] }
\Dx^2\derord[3]{u}{x} + \\
& \left[ \c\frac{\Dx^4}{5!} - \u\frac{\dDx^4}{5!}
\right]\derord[5]{u}{x} + \\
& \left[ \c\frac{\Dx^6}{7!} - \u\frac{\dDx^6}{7!}
\right]\derord[7]{u}{x} + \cdots
    \end{split}
\label{eq:o4}
\end{equation}

\begin{equation}
\def\u{\frac13}
\def\c{\frac43}
    \begin{split}
\c\frac{u_{j+1} - u_{j-1}}{2\Dx}-\u\frac{u_{j+2} -
u_{j-2}}{2\dDx} = & \derord{u}{x} + 
\left[ -\frac{1}{30} \right]\Dx^4\derord[5]{u}{x} + \\
& \left[ -\frac{1}{252} \right]\Dx^6\derord[7]{u}{x} + \cdots
    \end{split}
\label{eq:o4_s}
\end{equation}

\begin{equation}
	\therefore \epsilon = O\Dx^4
\end{equation}

\end{document}