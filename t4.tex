\documentclass{article}
\usepackage[utf8]{inputenc}
\usepackage[spanish,es-nodecimaldot,es-tabla]{babel}
\usepackage{amsmath}
\usepackage{graphicx}
\usepackage[colorlinks=true, allcolors=blue]{hyperref}
\usepackage[makeroom]{cancel}
% hyperref -->autoref, amsmath -->split
\usepackage{subfig,placeins}
\usepackage{libertine}
\usepackage[libertine]{newtxmath}
\graphicspath{{./figs/}{./imgs/}}
\usepackage[font=small,labelfont=bf]{caption}
\usepackage{listings,figs/tuneatantito}
\newcommand\pder[2]{\ensuremath {\dfrac{\partial#1}{\partial#2}}} 
\newcommand{\ppder}[2]{ \ensuremath {\dfrac{\partial^2 #1}{\partial #2^2}}}
\newcommand{\ppcder}[3]{ \ensuremath {\dfrac{\partial^2 #1}{\partial #2\partial #3}}}

\title{Tarea 4}
\author{\href{https://git.io/salvador}{Pedraza-Espitia S.}}
\date{}

\begin{document}

\maketitle

\section{Circulación atmosférica y circulación oceánica}
Componentes de los modelos

\section{Modelo de circulación atmosférica}

\section{Elípticas, parabólicas o hiperbólicas}

\section{... criterio de clasificación}

\section{Criterio de Courant-Friedrichs-Lewy}

\newcommand\derord[3][]{
\def\temp{#1}\ifx\temp\empty
	\ensuremath {\left( \dfrac{\mathrm{d}#2}{\mathrm{d}#3}} \right)_j%
\else
	\ensuremath {\left( \dfrac{\mathrm{d}^#1#2}{\mathrm{d}#3^#1}} \right)_j%
\fi}
%\newcommand\dero[3]{\ensuremath {\left( \dfrac{\mathrm{d}^#3#1}{\mathrm{d}#2^#3}} \right)_j}
\newcommand\Dt{\ensuremath {\left( \Delta t \right)}}
\newcommand\dxdero[3]{\ensuremath {\Dt^{\the\numexpr #3 - 1\relax} \derord[#3]{#1}{#2} }}

\section{Aproximación de la derivada en diferencias hacia adelante, hacia atrás y centradas}

\section{El esquema numérico de diferencias finitas hacia atrás es de orden 1}

%%%%%%%%
\begin{equation}
x_{j-1} = x_j - \Dt\derord{x}{t} + \frac12\Dt^2\derord[2]{u}{t} - \frac1{3!}\Dt^3\derord[3]{u}{t} + \cdots
\end{equation}

\begin{equation}
\frac{x_{j-1} - x_j}{\Dt} = - \derord{x}{t} + \frac12\Dt\derord[2]{u}{t} - 
	\frac1{3!}\dxdero{u}{t}{3} + \cdots
\end{equation}

\begin{equation}
\frac{x_{j} - x_{j-1}}{\Dt} = \derord{x}{t} - \frac12\Dt\derord[2]{u}{t} + \frac1{3!}\dxdero{u}{t}{3} - \cdots
\end{equation}

\begin{equation}
\epsilon = - \frac12\Dt\derord[2]{u}{t} + \frac1{3!}\dxdero{u}{t}{3} - \cdots
\end{equation}

\begin{equation}
\epsilon = O\Dt
\end{equation}
%%%%%%%%%

\section{Ecuación de advección con el esquema de diferencias centradas}
%\lstinputlisting[language=Matlab]{./MatlabCodes/t4inciso9.m}
%\input{inputs/t4i9}

\section{Esquema de salto de rana y estabilidad}

\section{Esquema de diferencias centradas con \texorpdfstring{$O(\Delta x)$}{} error de orden 4}

\end{document}
